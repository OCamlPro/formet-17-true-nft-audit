
The implementation of NFT studied in this document is a Proof of Concept
of how NFT primitives should be built on FreeTON.
%
It does not correspond to the actual implementation of an NFT that would
be directly deployed.
%
Instead, these contracts act as templates of how NFT contracts should be implemented to provide the same interface as other NFTs on FreeTON.
%
As such, the different contracts of TrueNFT-core should be modified, or at minimum considered as just a source of inspiration.
%
Considering this special aspect of these contracts, serving as a 
\emph{specification written in Solidity}, we focused this audit on two 
aspects.
%
First, we looked for issues in the NFT-Core logic, i.e. how the different
contracts and data interactions were implemented with respect to what an NFT
should be.
%
Second, as a usual audit, we searched for actual issues in the code.
%
While this second kind of issues may not be relevant (as the project is a 
POC that should be modified), we assumed that if this repository is expected
to be a reference, it should be perfect in every regard. 

\section{Specification}

Non Fungible Tokens are deployed from a \textbf{NftRoot} contract which will
be the basis of all minted tokens.
%
This contract has two purposes.
\begin{itemize}
    \item The deployment of a \textbf{Data} contract, representing a
    fraction of the digitalized asset (i.e. the NFT itself). So as to
    easily retrieve the information of the asset from outside the
    blockchain, the \textbf{Data} contract deploys two \textbf{Index}
    contracts, each pointing to the \textbf{Data} address: one
    retrievable from the NftRoot address and the owner's address
    (allowing users to list all NFTs derived from this root and owned
    by the same owner), and one from only the owner's address
    (allowing users to list all NFTs owner by the same owner).

\item The deployment of an \textbf{IndexBasis} contract with the Data code hash.
\end{itemize}

These four contracts represent the whole NFT core implementation.
%
The DeBot SDK provides a primitive to list all contracts with a given
code-hash. To use this primitive to list NFTs, the contracts make use
of {\em salted} codes: for example, \textbf{Index} contracts are
deployed with the same code, but salted with the owner's address, and
either the \textbf{NftRoot} address or zero, so as to create different
code-hashes (with actually the same code!). As a consequence, the
DeBot primitive can be used to list all NFTs for a given owner
address, within or not a specific NFT root.

From our audit, we think that this mechanism is safe, and works as
expected.

\section{Generic issues}

Before reading in detail the source code, several issues (mostly coding habits) affected the
project as a whole. We list them in this section.

\issueMajor{Funds accessibility and bounced messages}{
    Unless a contract is destroyed (which is not the case for all
    contracts), funds are not accessible.  As there is no error
    handling, especially for bounced messages, funds may accumulate on
    the contracts. However, there are no provided functions to recover
    such funds to the user.  }

%\issueMinor{Unsafe assumptions of message origin}{
%    The messages are assumed to be from a contract; the case
%    msg.sender == address(0) is never treated.  }

\issueMinor{Naming convention}{
    Static variables should start with a prefix like "s\_" and globals
    should start with a prefix like "g\_" or "m\_" and
    internal/private functions should start with "\_". Following such
    rules would make these contracts much easier to read and audit.  }
