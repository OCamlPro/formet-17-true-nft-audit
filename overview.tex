\section{Specification}

Non Fungible Tokens are deployed from a \textbf{NftRoot} contract which will
be the basis of all minted tokens.
%
This contract has two purposes.
\begin{itemize}
    \item The deployment of a \textbf{Data} contract, representing a fraction of the
    digitalized asset. So as to easily retrieve the information of the asset, 
    it deploys two \textbf{Index} contracts: one retrievable from the NftRoot address,
    the owner's address and the Data address, and one from the NftRoot address
    and the owner's only. 
    \item The deployment of an \textbf{IndexBasis} contract with the Data code hash.
\end{itemize}

These four contracts represent the whole NFT core implementation.
%
Many interesting features of this infrastructure comes from the use of salted code
for generating specific code hashes that can be found with specific instructions accessible 
by DeBots.
%
This audit does not discuss the retrivability of the contracts, but our study of the TrueNFT 
functioning did not show critical issues in this regard.

\section{Generic issues}

Before reading in detail the source code, several issues (mostly coding habits) affected the
project as a whole. We list them in this section.

\issueMajor{Funds accessibility and bounced messages}{
    Unless a contract is destroyed (which is not the case for all contracts), 
    funds are not accessible.
    %
    There is no error handling, especially for bounced messages, which allow
    funds to be stored on contracts.
}

\issueMinor{Unsafe assumptions of message origin}{
    The messages are assumed to be from a contract ; the case msg.sender == address(0)
    is never treated.
}

\issueMinor{Naming convention}{
    Static variables should start with a prefix like "s\_" and
    globals should start with a prefix like "g\_" or "m\_"
    and internal/private functions should start with "\_".
}
